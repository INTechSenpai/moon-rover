\documentclass[a4paper, 12pt]{article}

\usepackage[utf8]{inputenc}
\usepackage[T1]{fontenc}
\usepackage[francais,english]{babel}
\usepackage{hyperref}
\title{Ordres entre le bas niveau et le haut niveau}
 
\author{Moon-rover}

\date{}
\setlength\parindent{0pt} 
\begin{document}

\selectlanguage{francais}
\maketitle

\pagestyle{plain}

\tableofcontents


\section{Introduction}

Ces ordres sont indépendants de la couche protocole, qui est plus basse.
Ils prennent place dans le champ "données" dudit protocole.

La structure d'un ordre est la suivante : un code sur un octet, puis des données. La taille de chaque donnée est donnée en bit à côté de son nom.

\section{Haut niveau vers bas niveau}

\subsection{Utilise actionneurs}
\texttt{0x06 ID(8) ANGLE(16)}

L'ID est l'ID de l'AX-12.

L'angle est l'angle à fournir à l'AX-12 (entre 0 et 300).

\subsection{Arrêt du robot}
\texttt{0x07}
   
Le robot doit s'arrêter (asservissement sur place). Le bas niveau doit arrêter les ordres longs en cours en rapport avec les déplacements.

\subsection{Avancer}

\subsubsection{Avancer, avec une distance positive, avec une vitesse de translation maximale}
    \texttt{0x0C DISTANCE(16) VITESSE  TRANSLATION  MAX(16)}

\subsubsection{Avancer, avec une distance négative, avec une vitesse de translation maximale}

    \texttt{0x0D |DISTANCE|(16) VITESSE  TRANSLATION  MAX(16)}
    
\subsubsection{Avancer dans la même direction que précédemment}

    \texttt{0x0E DISTANCE(16) VITESSE  TRANSLATION  MAX(16)}
    
Probablement pas utilisé.

\subsubsection{Avancer dans la direction opposée à précédemment}

    \texttt{0x0F DISTANCE(16) VITESSE  TRANSLATION  MAX(16)}

Utilisé pour se dégager.

\subsection{Désasservissement}
    \texttt{0x28}

Le robot doit se désasservir.

\subsection{Trajectoire courbe}

Remarque : le dernier bit de l'octet d'ordre permet de savoir s'il faut ou non effacer tous les arcs reçus.

En cas de marche arrière, il sera envoyé l'orientation voulue du robot + $2\pi$. Autrement dit, l'orientation envoyée sera toujours l'orientation vers laquelle avance le robot.

\subsubsection{Ajoute un arc de TC}

    \texttt{0x2E X+ POINT(12) Y POINT(12) ORIENTATION(16) COURBURE(16) VITESSE  TRANSLATION(16)}
    
\subsubsection{Ajoute arc trajectoire courbe précédé d'un arrêt}

    \texttt{0x2C X+ POINT(12) Y POINT(12) ORIENTATION(16) COURBURE(16) VITESSE TRANSLATION(16)}

\subsubsection{Efface tous les arcs reçus et rajoute cet arc}

    \texttt{0x2F X+ POINT(12) Y POINT(12) ORIENTATION(16) COURBURE(16) VITESSE TRANSLATION(16)}
    
\subsubsection{Efface tous les arcs reçus et rajoute cet arc, précédé d'un arrêt} 

\texttt{0x2D X+ POINT(12) Y POINT(12) ORIENTATION(16) COURBURE(16) VITESSE TRANSLATION(16)}
   
\subsection{Demande notification début match}
    \texttt{0x11}

Le bas niveau préviendra du début du match.
   
   \subsection{Demande notification fin match}
       \texttt{0x12}
   
   Le bas niveau préviendra de la fin du match.
   
\subsection{Demande couleur}
    \texttt{0x13}

Le bas niveau doit renvoyer la couleur du robot pour ce match.

    
\subsection{Start capteurs}
\texttt{0x10}
    
Le bas niveau doit démarrer le stream des capteurs.
    
\subsection{Ping}
    \texttt{0x5A}
    
    Parce que je suis un 5A et que c'est moi qui donne les valeurs.
    
    Une réponse "pong" est attendue.

\section{Bas niveau vers haut niveau}

\subsection{Robot arrivé}
    \texttt{0x02}
    
Le mouvement est terminé, le robot est arrivé à destination.
    
\subsection{Problème mécanique}
    \texttt{0x03}
    
Le robot n'a pas pu finir son mouvement à cause d'une erreur mécanique (collision).

\subsection{Arrêt demandé par le haut niveau}
    \texttt{0x01}
    
Lorsque le haut niveau demande une immobilisation du robot, le bas niveau doit arrêter les ordres longs en cours en rapport avec les déplacements et fournir ce code.
    
\subsection{Début match}
    \texttt{0x04}

Le match vient de commencer.
    
\subsection{Match fini}
    \texttt{0x05}
    
Le match vient de finir.
    
\subsection{Couleur robot}
\subsubsection{Côté gauche}
\texttt{0x06}
    
Côté gauche de la table (côté arbre-sur-lequel-les-gens-bourrés-grimpent dans le local INTech)
\subsubsection{Côté droit}
\texttt{0x07}
    
Côté droite de la table (côté Minet dans le local INTech)
\subsubsection{Inconnue}
\texttt{0x08}
    
La couleur n'est pas encore connue
    
\subsection{Stream capteurs}

Le bas niveau ne doit commencer à streamer ces valeurs qu'après en avoir reçu l'ordre.

Les capteurs renvoient chacun une distance en $mm$.

\subsubsection{Le robot est en marche avant}
    \texttt{0x0A X+(12) Y(12) ORIENTATION(16) COURBURE(8) VIT  LIN(16) VIT  ROT(16) CPT1(12) CPT2(12) …}
    
\subsubsection{Le robot est en marche arrière}
    \texttt{0x0B X+(12) Y(12) ORIENTATION(16) COURBURE(8) VIT  LIN(16) VIT  ROT(16) CPT1(12) CPT2(12) …}

\subsection{Pong}
     \texttt{0x0C}

Doit être envoyé après réception d'un ping.

\section{Codage des valeurs}

\subsection{Endianness}

Les valeurs sur plusieurs octets sont écrites en \textbf{big endian} : l'octet de poids fort en tête.

\subsection{Valeurs non signées}

Toutes les valeurs sont envoyées non-signées, ce qui signifie qu'elles sont toutes positives.

\subsection{Repère}

Le repère utilisé est le repère classique d'INTech : l'axe X est le côté long de la table, et va de -1500mm à +1500mm. L'axe Y est le côté court de la table, et va de 0mm à 2000mm.

L'orientation est toujours donnée de manière absolue, par rapport au cercle trigonométrique usuelle (le 0 est dans la direction de l'axe X positif)

\subsection{Unités}

Sauf mention contraires, les unités utilisés par défaut sont le $mm$ pour les distances, la $ms$ pour les durée et le $mrad$ pour les angles.

Exceptions :
\begin{itemize}
\item La courbure est toujours envoyée en $m^{-1}$.
\item Les vitesses linéaires sont transmises en $mm \cdot s^{-1}$.
\item Les vitesses de rotation sont transmises en $mrad \cdot s^{-1}$.
\end{itemize}

Ces unités ont été toutes choisies afin de pouvoir envoyer les valeurs comme des entiers représentables sur un nombre limités de bits.

\subsection{Codage d'une position}

Les positions (x,y) sont toujours envoyées sur 3 octets, X puis Y (chacun sur 12 bits). Il est important que remarquer que puisque la valeur de X à envoyer doit être positive, il faudra envoyer la valeur X+1500. C'est pour cela que dans les ordres, il est écrit X+.

\end{document}